% !TEX root = ../diploma.tex

\section{Код и псевдокод}
Вставить код тоже просто. Если настройки листинга не устраивают, их можно изменить. Макрос настройки находится в файле \texttt{commands}.

\begin{lstlisting}[caption={Пример вызова БПФ в библиотеке \texttt{CuFFT}}]
    cufftComplex *d_signal;
    cudaMalloc((void **) &d_signal, mem_size);
    cudaMemcpy(d_signal, fg, mem_size, cudaMemcpyHostToDevice);

    cufftHandle plan;
    cufftPlan2d(&plan, N, N, CUFFT_C2C);

    cufftExecC2C(plan, (cufftComplex *)d_signal, (cufftComplex *)d_signal, CUFFT_FORWARD);
\end{lstlisting}

Также можно писать псевдокод. Ключевые слова можно переводить, вводить новые конструкции и т.д. Пример в файле \texttt{commands}.
\begin{algorithm}[H]
    \caption{Пример псевдокода}
    \begin{algorithmic}[1] % The number tells where the line numbering should start
        \Procedure{F}{$A$, $B$, $N$}
        \State $E \gets A$
        \For{$i := 1$ до $N$}
        \State $\hat{E} = \texttt{fft}~E$
        \EndFor
        \State \textbf{вернуть} $E$
        \EndProcedure
    \end{algorithmic}
\end{algorithm}
