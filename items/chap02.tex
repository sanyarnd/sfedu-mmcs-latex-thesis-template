% !TEX root = ../diploma.tex

\section{Списки}

Существует 3 базовых окружения для списков.

Маркированные списки:
\begin{itemize}
    \item пункт 1
    \item пункт 2
\end{itemize}

Нумерованные списки:
\begin{enumerate}
    \item пункт 1
    \item пункт 2
\end{enumerate}

Описания:
\begin{description}
    \item[Пункт] 1
    \item[Пункт] 2
\end{description}

Иногда хочется сжать список. Чтобы не настраивать интервал между списками (это делается не очень удобно) достаточно передать параметр \texttt{[noitemsep]}. 
Этот параметр (а также ряд других, которые можно посмотреть в документации) можно задать глобально после подключения пакета \texttt{enumitem}: \texttt{\backslash setlist[itemize]\{noitemsep, nolistsep\}}.


Без сжатия:
\begin{itemize}
    \item пункт 1
    \item пункт 2
\end{itemize}

Со сжатием:
\begin{itemize}[noitemsep]
    \item пункт 1
    \item пункт 2
\end{itemize}
