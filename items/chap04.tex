% !TEX root = ../diploma.tex

\section{Таблицы}
Здесь используется вспомогательное окружение \texttt{tabularx}, которое управляет шириной столбцов и автоматически переносит текст на новую строку в той же ячейке при нехватке размерности, доступной тексту.

\begin{table}[H]
    \centering
    \begin{tabularx}{\textwidth}{| X | X | X |}
    	\hline
    	\textbf{Размер изображения} & \textbf{Время GPU} & \textbf{Время CPU} \\ \hline
    	$1920\times 1920$           & $6$ мс             & $75$ мс            \\ \hline
    	$4096\times 4096$           & $24$ мс            & $520$ мс           \\ \hline
    	$3648\times 5472$           & $35$ мс            & $625$ мс           \\ \hline
    \end{tabularx}
    \caption{Сравнение скорости работы}
\end{table}

Таблицы как выше просты в использовании, но их внешний вид оставляет желать лучшего. При оформлении работ можно руководствоваться следующим пособием:
\begin{center}
    \href{https://www.inf.ethz.ch/personal/markusp/teaching/guides/guide-tables.pdf}{https://www.inf.ethz.ch/personal/markusp/teaching/guides/guide-tables.pdf}
\end{center}

Пример красивой таблицы:
\begin{table}[H]
    \centering
    \caption{Подпись к таблице}
    \begin{tabular}{llr}
    	\toprule
    	          \multicolumn{2}{c}{Item}            &           \\
    	\cmidrule(r){1-2}
        Животное & Описание & Цена (\$) \\ \midrule
    	Armadillo                          & frozen   &      8.99 \\ \bottomrule
    \end{tabular}
\end{table}
