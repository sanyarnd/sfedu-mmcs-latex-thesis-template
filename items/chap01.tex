% !TEX root = ../diploma.tex

\section{Цитирование и ссылки}
Делать ссылки к библиографии несложно: достаточно поставить \texttt{\\cite\{ссылка\}}. Ссылки не пишутся слитно, поэтому перед \texttt{cite} нужен пробел. Выглядит это примерно так \cite{test}.

Для уравнений можно использовать специальное окружение. Любые окружения (в т.ч. уравнения) можно помечать, чтобы в дальнейшем иметь возможность поставить ссылку. Для этого используется \texttt{label}. Для уравнений есть специальная (совсем необязательная) версия: \texttt{label\{eq:имя\}} Пример:
\begin{equation}\label{eq:some_eq}
    e^2 = E\{(F - Y)^2\},
\end{equation}
где $E$ -- математическое ожидание.

Чтобы получить ссылку достаточно вставить макрос \texttt{ref\{имя\}}. Для уравнений (в случае использования специальной версии) есть \texttt{eqref}. Получим следующее: \eqref{eq:some_eq}.
