% !TEX root = ../diploma.tex

\section{Цитирование и ссылки}
В \LaTeX~присутствует два механизма указания ссылок: это метки и цитирование. 

\paragraph{Ссылки} указываются на произвольную метку, которую задает сам пользователь. Для установки используется \verb+\label{<маркер>:<имя>}+. Ссылками можно помечать заголовки и любые окружения. 
Маркер совершенно не обязательно указывать, но его наличие сильно упрощает управление и поиск ссылок. 

Ссылаться на метку можно при помощи команды \verb+\ref{<имя>}+.

Помимо приведенного выше, существует несколько других вариантов команд ссылок. Одним из таких специальных вариантов является \verb+\eqref{<имя>}+. Разница выглядит следующим образом:
\begin{equation}\label{eq:1}
    a^2+b^2=c^2, 
\end{equation}
\verb+\eqref{eq:1} и \ref{eq:1}+: \eqref{eq:1} и \ref{eq:1}.

Существуют и другие варианты, например, \verb+\autoref{eq:1}+: \autoref{eq:1}.

\paragraph{Цитирование} происходит на элемент в библиографии. Для цитирования используется команда \verb+\cite{<ссылка>}+. 

Перед \verb+\cite+ или \verb+\ref+ обычно ставится пробел. Выглядит это так: \cite{test}.

Как и со ссылками, существуют отличные от \verb+\cite+ команды цитирования. Их достаточно много, с кучей тонкостей, и обсуждение вопросов оформления цитирования выходит за рамки обзора возможностей \LaTeX.


Стандартным решением, чтобы убрать "лишнюю" нумерацию с фигур/уравнений и проч. нумеруемых сущностей, является пакет 
\verb+chngcntr+. После его подключения можно воспользоваться командой \verb+\counterwithout{figure}{chapter}+, где вместо figure может быть любое требуемое окружение.
