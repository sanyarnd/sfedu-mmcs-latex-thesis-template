% !TEX root = ../diploma.tex

\section{Цитирование и ссылки}
В \LaTeX~присутствует два механизма указания ссылок: это метки и цитирование. 

\paragraph{Ссылки} указываются на произвольную метку, который задает сам пользователь. Для установки используется \verb+\label{<маркер>:<имя>}+. Ссылками можно помечать заголовки и любые окружения. 

Маркер совершенно не обязательно указывать, но его наличие сильно упрощает управление и поиск ссылок. 

Ссылаться на метку можно при помощи команды \verb+\ref{<имя>}+.

Существует несколько вариантов получения ссылки. Одним из таких специальных вариантов является \verb+\eqref{<имя>}+. Разница выглядит следующим образом:
\begin{equation}\label{eq:1}
    a^2+b^2=c^2, 
\end{equation}
\verb+\eqref{eq:1} и \ref{eq:1}+: \eqref{eq:1} и \ref{eq:1}.

Существуют и другие варианты, например, \verb+\autoref{eq:1}+: \autoref{eq:1}.

\paragraph{Цитирование} происходит на элемент в библиографии. Для цитирования используется следующая команда: \verb+\cite{<ссылка>}+. 

Любые ссылки не пишутся слитно, поэтому перед \verb+\cite+ нужен пробел. Выглядит это так: \cite{test}.

Как и со ссылками, существуют отличные от \verb+\cite+ команды цитирования. Их достаточно много, с кучей тонкостей, и обсуждение вопросов оформления цитирования выходит за рамки обзора возможностей \LaTeX.


