% !TEX root = ../diploma.tex

\section{Фигуры}
В окружение \texttt{figure} можно помещать обычный \texttt{includegraphics}, таблицы, элементы \texttt{tikz}, создавать массивы изображений и т.д.

\paragraph{Пример массива изображений.}
Подписи не обязательны. Нумерацию \\ \texttt{subfloat}'ов можно выключить в \texttt{captionsetup}. Там же находится набор других настроек внешнего вида подписей.

Расстояние между картинками задается стандартными макросами шага: \texttt{quad, qquad} и т.д.

Если картинки нет, но необходимо уже сейчас отрегулировать внешний вид и размер, то можно использовать стандартные \texttt{example-image-[a,b,c]}.

\begin{figure}[H]
    \centering
    \captionsetup[subfigure]{justification=centering}
    \subfloat[Пример]{\includegraphics[height=0.1\textwidth]{example-image-a}}%
    \quad
    \subfloat[Пример]{\includegraphics[height=0.15\textwidth]{example-image-b}}%
    \quad
    \subfloat[Пример]{\includegraphics[height=0.1\textwidth]{example-image-c}}%
    \quad
    \subfloat[Очень длинная подпись, которая перенесется на новую строку]{\includegraphics[height=0.15\textwidth]{example-image-a}}%
    \caption{Общая подпись к фигуре}
\end{figure}
