% !TEX root = diploma.tex

%% -------------------------
%% Окружения для алгоритмов
%% -------------------------
\usepackage[]{algorithm}
\usepackage[]{algpseudocode}


%% --------------------
%% Листинги кода
%% --------------------
\usepackage[]{listings}


%% ----------------
%% Таблицы и списки
%% ----------------
\usepackage[]{enumitem}
\usepackage[]{tabularx}
\usepackage[]{booktabs}
\usepackage[]{multicol}
\usepackage[]{longtable}


%% ----------------------------
%% Приложение
%% ----------------------------
\usepackage[titletoc]{appendix}


%% ----------------------------------
%% Работа с изображениями и графиками
%% ----------------------------------
\usepackage[]{graphicx}
\usepackage[]{subfig}
\usepackage[]{float}

\usepackage[]{tikz}
\usepackage[]{pgfplots}
\pgfplotsset{compat=newest}


%% --------------------
%% Загрузка данных
%% --------------------
\usepackage[]{datatool}

%% -----------------------
%% Единицы изменения
%% -----------------------
\usepackage[
    binary-units = true, 
    per-mode     = symbol
]{siunitx}



%% --------------------------------------
%% Показать границы страницы, удобно для
%% проверки нарушения размерности
%% для активации просто раскомментировать
%% --------------------------------------
%\usepackage[]{showframe}


%% --------------------------
%% Использование языка Python
%% --------------------------
%\usepackage[]{pythontex}
