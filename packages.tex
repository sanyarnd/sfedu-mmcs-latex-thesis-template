% !TEX root = diploma.tex

%% ----------------------
%% Математические пакеты
%% ----------------------
\usepackage[shortcuts]{extdash} % Дополнительные команды для различных дефисов, тире и пр.
\usepackage[]{amsthm}           % окружения теоремы и пр.
\usepackage[]{amsfonts}         % поддержка ажурных шрифтов и пр.
\usepackage[]{amscd}            % коммутативные диаграммы
\usepackage[]{mathtools}        % доп. функции

%% -------------------------
%% Окружения для алгоритмов
%% -------------------------
\usepackage[]{algorithm}
\usepackage[]{algpseudocode}

%% --------------------
%% Листинги кода
%% --------------------
\usepackage[]{listings}

%% ----------------
%% Таблицы и списки
%% ----------------
\usepackage[]{enumitem}
\usepackage[]{tabularx}
\usepackage[]{tabulary}
\usepackage[]{multicol}
\usepackage[]{multirow}
\usepackage[]{booktabs}

%% ----------------------------
%% Приложение
%% ----------------------------
\usepackage[titletoc]{appendix}

%% ----------------------------------
%% Работа с изображениями и графиками
%% ----------------------------------
\usepackage[]{graphicx}
\usepackage[]{subfig}
\usepackage[]{float}

\usepackage[]{tikz}
\usepackage[]{pgfplots}
\pgfplotsset{compat=newest}

%% --------------------
%% Загрузка данных
%% --------------------
\usepackage[]{datatool}

%% -----------------------
%% Единицы изменения
%% -----------------------
\usepackage[
    binary-units = true, 
    per-mode     = symbol
]{siunitx}

%% --------------------------------------
%% Показать границы страницы, удобно для
%% проверки нарушения размерности
%% --------------------------------------
%\usepackage[]{showframe}

%% --------------------------
%% Использование языка Python
%% --------------------------
%\usepackage[]{pythontex}

%% ----------------
%% Ссылки
%% ----------------
% Обычно загружается последним, если не сказано 
% обратного в документации конфликтующего пакета
\usepackage[
    unicode,
    colorlinks = true,
    urlcolor   = black,
    linkcolor  = black,
    filecolor  = black,
    citecolor  = black,
    pdfkeywords={thesis}
]{hyperref}

%% -----------------------------
%% Ускорение работы для Overleaf
%% -----------------------------
\makeatletter
  \@ifpackagelater{microtype}{2017/07/05}{}{%
    \patchcmd{\MT@is@composite}
      {\expandafter\expandafter\expandafter}
      {\ifx\UnicodeEncodingName\@undefined\else
      \expandafter\expandafter\expandafter\MT@is@uni@comp\MT@char\iffontchar\else\fi\relax
    \fi\expandafter\expandafter\expandafter}
      {}{}
    \def\MT@is@uni@comp#1\iffontchar#2\else#3\fi\relax{%
      \ifx\\#2\\\else\edef\MT@char{\iffontchar#2\fi}\fi
    }
  }
\makeatother

