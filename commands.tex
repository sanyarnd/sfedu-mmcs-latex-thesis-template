% !TEX root = diploma.tex

%% ----------------------
%% Стандартные каталоги
%% ----------------------
\graphicspath{{images/}}
\def\datapath{data/}


%% ------------------------
%% Файл с библиографией
%% ------------------------
\addbibresource{biblio.bib}


%% ---------------------------
%% Параметры для листинга кода
%% ---------------------------
\renewcommand{\lstlistingname}{Листинг}
\lstset{
    language        = [ISO]C++,
    basicstyle      = \footnotesize\ttfamily,
    keywordstyle    = \footnotesize\ttfamily\bfseries,
    stringstyle     = \footnotesize\ttfamily,
    commentstyle    = \footnotesize\textit,
    breaklines      = true,
    tabsize         = 2,
    literate        = {\ \ }{{\ }}1,
    showstringspaces= false,
    inputencoding   = utf8x,
    numbers         = left,
    stepnumber      = 1,
    numbersep       = 5pt,
    backgroundcolor = \color{white},
    frame           = tb
}


%% -----------------------------------
%% Переименование теорем, аксиом и пр.
%% -----------------------------------
\newtheorem{theorem}{Теорема}[section]
\newtheorem{axiom}{Аксиома}[section]
\newtheorem{corollary}{Следствие}[theorem]
\newtheorem{lemma}{Лемма}[section]

\theoremstyle{definition}
\newtheorem{definition}{Определение}[section]

% Черный квадрат в конце доказательства
\renewcommand{\qedsymbol}{\ensuremath{\blacksquare}}


%% ------------
%% Мат. символы
%% ------------
\renewcommand{\epsilon}{\ensuremath{\varepsilon}}
\renewcommand{\phi}{\ensuremath{\varphi}}
\renewcommand{\kappa}{\ensuremath{\varkappa}}
\renewcommand{\le}{\ensuremath{\leqslant}}
\renewcommand{\leq}{\ensuremath{\leqslant}}
\renewcommand{\ge}{\ensuremath{\geqslant}}
\renewcommand{\geq}{\ensuremath{\geqslant}}
\renewcommand{\emptyset}{\varnothing}


%% --------------
%% Для алгоритмов
%% --------------
\makeatletter
\renewcommand{\ALG@name}{Алгоритм}
%\renewcommand{\listalgorithmname}{Список \ALG@name ов}
\makeatother

\renewcommand{\algorithmicprocedure}{\textbf{процедура}}
\renewcommand{\algorithmicrequire}{\textbf{Требуется:}}
\renewcommand{\algorithmicensure}{\textbf{Проверить:}}
\renewcommand{\algorithmiccomment}[1]{\{#1\}}
\renewcommand{\algorithmicend}{\textbf{конец}}
\renewcommand{\algorithmicif}{\textbf{if}}
\renewcommand{\algorithmicthen}{\textbf{then}}
\renewcommand{\algorithmicelse}{\textbf{else}}
\renewcommand{\algorithmicfor}{\textbf{для}}
\renewcommand{\algorithmicforall}{\textbf{для всех}}
\renewcommand{\algorithmicdo}{\textbf{делать}}
\renewcommand{\algorithmicwhile}{\textbf{пока}}
\newcommand{\algorithmicelsif}{\algorithmicelse\ \algorithmicif}
\newcommand{\algorithmicendif}{\algorithmicend\ \algorithmicif}
\newcommand{\algorithmicendfor}{\algorithmicend\ \algorithmicfor}
