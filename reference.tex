\documentclass[a4paper, 14pt, oneside, DIV]{scrartcl}

%% Взято из .cls файла

\usepackage[]{fontspec}         % загрузка шрифтов, работа с кодировкой и пр.
\usepackage[]{polyglossia}      % переносы слов
\usepackage[]{microtype}        % различные исправления типографии
\usepackage[]{indentfirst}      % красная строка в первом абзаце

\usepackage
[	onehalfspacing,
    nodisplayskipstretch
]{setspace} 					% полуторный интервал

\setlength{\parindent}{1.25cm}  %% Красная строка
\frenchspacing                  %% убираем лишние отступы после точек

%% Перенос слов
\binoppenalty       = 10000 %% Запрет переносов строк в формулах
\relpenalty         = 10000
\pretolerance       = 5000  %% Настройки переноса
\tolerance          = 9000  %% Настройки переноса
\emergencystretch   = 0pt   %% Запрещаем выход за границы
\righthyphenmin     = 2     %% целое число, равное наименьшему количеству букв в слове, которые можно переносить на следующую строку
\lefthyphenmin      = 2
\hyphenpenalty      = 500
\clubpenalty        = 10000 %% Запрет разрывов страниц после первой
\widowpenalty       = 10000 %% и перед предпоследней строкой абзаца

%% Язык
\setmainlanguage[spelling=modern]{russian}
\setotherlanguage{english}

%% Шрифты
\defaultfontfeatures{Ligatures = {TeX, Common}, Mapping = tex-text}

%% Main font
\setmainfont
[   Path            = fonts/,
    Extension       = .otf,
    UprightFont     = *-Regular,
    BoldFont        = *-Bold,
    ItalicFont      = *-Italic,
    BoldItalicFont  = *-BoldItalic
]{FreeSerif}
\newfontfamily\cyrillicfont
[   Path            = fonts/,
    Extension       = .otf,
    UprightFont     = *-Regular,
    BoldFont        = *-Bold,
    ItalicFont      = *-Italic,
    BoldItalicFont  = *-BoldItalic
]{FreeSerif}

\pagestyle{empty} % убираем нумерацию

\begin{document}
    \noindent\makebox[\textwidth][r]{
    \begin{minipage}{.5\textwidth}\raggedleft
        \underline{Данные организации}\\
        \today
    \end{minipage}
    }

    \begin{center}
        \textbf{Отзыв о прохождении преддипломной практики}
    \end{center}

    В период с xx.xx.20xx г. по xx.xx.20xx г. \underline{Ф.И.О.} проходил~(-а) преддипломную практику в \underline{место прохождения практики}.

    За время преддипломной производственной практики \underline{И.О.} изучил~(-а) \underline{список} \underline{приобретенных навыков}.

    \underline{И.О.} показал~(-а) себя подготовленным специалистом, правильно применял~(-а) полученные при обучении в университете знания.

    \underline{И.О.} работает производительно, достигает хороших результатов в процессе выполнения заданий. Стремится к повышению своей квалификации, используя для этого все имеющиеся возможности и осуществляя поиск новой информации.

    Считаю, что \underline{Ф.И.О.}, в ходе прохождения практики, как специалист, продемонстрировал~(-а) высокий уровень подготовки и по результатам проделанной работы заслуживает оценки \underline{<<оценка>>}.
    \\\\

    \noindent \hspace{15mm} \underline{Должность} \hspace{100mm} \underline{Ф.И.О.}
\end{document}
