%% ----------------------------------------
%% Выбрать необходимый стиль
%% ----------------------------------------
%% scrartcl для бакалавров, scrbook для магистров
\documentclass[scrartcl]{sfedu-mmcs-thesis}

% !TEX root = diploma.tex

%% ---------------------
%% Математические пакеты
%% ---------------------
\usepackage[]{amsthm}
\usepackage[]{amsfonts}
\usepackage[]{amsmath}
\usepackage[]{amssymb}
\usepackage[]{amscd}
\usepackage[]{mathtools}
\usepackage[]{unicode-math}
\usepackage[]{algorithm}
\usepackage[]{algpseudocode}

%% --------------------
%% Листинги
%% --------------------
\usepackage[]{listings}

%% --------------------
%% Таблицы и списки
%% --------------------
\usepackage[]{enumitem}
\usepackage[]{tabularx}

%% ----------------------------
%% Приложение
%% ----------------------------
\usepackage[titletoc]{appendix}

%% ----------------------
%% Работа с изображениями
%% ----------------------
\usepackage[]{graphicx}
\usepackage[]{subfig}
\usepackage[]{float}
% !TEX root = diploma.tex

%% -----------------------------------
%% Стандартный каталог для изображений
%% -----------------------------------
\graphicspath{{images/}}


%% ------------------------
%% Файл с библиографией
%% ------------------------
\addbibresource{biblio.bib}


%% ---------------------------
%% Параметры для листинга кода
%% ---------------------------
\renewcommand{\lstlistingname}{Листинг}
\lstset{
    language        = [ISO]C++,
    basicstyle      = \footnotesize\ttfamily,
    keywordstyle    = \footnotesize\ttfamily\bfseries,
    stringstyle     = \footnotesize\ttfamily,
    commentstyle    = \footnotesize\textit,
    breaklines      = true,
    tabsize         = 2,
    literate        = {\ \ }{{\ }}1,
    showstringspaces= false,
    inputencoding   = utf8x,
    numbers         = left,
    stepnumber      = 1,
    numbersep       = 5pt,
    backgroundcolor = \color{white},
    frame           = tb
}


%% -----------------------------------
%% Переименование теорем, аксиом и пр.
%% -----------------------------------
\newtheorem{theorem}{Теорема}[section]
\newtheorem{axiom}{Аксиома}[section]
\newtheorem{corollary}{Следствие}[theorem]
\newtheorem{lemma}{Лемма}[section]

\theoremstyle{definition}
\newtheorem{definition}{Определение}[section]

% Черный квадрат в конце доказательства
\renewcommand{\qedsymbol}{\ensuremath{\blacksquare}}


%% ------------
%% Мат. символы
%% ------------
\renewcommand{\epsilon}{\ensuremath{\varepsilon}}
\renewcommand{\phi}{\ensuremath{\varphi}}
\renewcommand{\kappa}{\ensuremath{\varkappa}}
\renewcommand{\le}{\ensuremath{\leqslant}}
\renewcommand{\leq}{\ensuremath{\leqslant}}
\renewcommand{\ge}{\ensuremath{\geqslant}}
\renewcommand{\geq}{\ensuremath{\geqslant}}
\renewcommand{\emptyset}{\varnothing}


%% --------------
%% Для алгоритмов
%% --------------
\makeatletter
\renewcommand{\ALG@name}{Алгоритм}
%\renewcommand{\listalgorithmname}{Список \ALG@name ов}
\makeatother

\renewcommand{\algorithmicprocedure}{\textbf{процедура}}
\renewcommand{\algorithmicrequire}{\textbf{Требуется:}}
\renewcommand{\algorithmicensure}{\textbf{Проверить:}}
\renewcommand{\algorithmiccomment}[1]{\{#1\}}
\renewcommand{\algorithmicend}{\textbf{конец}}
\renewcommand{\algorithmicif}{\textbf{if}}
\renewcommand{\algorithmicthen}{\textbf{then}}
\renewcommand{\algorithmicelse}{\textbf{else}}
\renewcommand{\algorithmicfor}{\textbf{для}}
\renewcommand{\algorithmicforall}{\textbf{для всех}}
\renewcommand{\algorithmicdo}{\textbf{делать}}
\renewcommand{\algorithmicwhile}{\textbf{пока}}
\newcommand{\algorithmicelsif}{\algorithmicelse\ \algorithmicif}
\newcommand{\algorithmicendif}{\algorithmicend\ \algorithmicif}
\newcommand{\algorithmicendfor}{\algorithmicend\ \algorithmicfor}

% !TEX root = diploma.tex

%% Math font
\setmathfont[
    Path        = fonts/,
    Extension   = .otf,
    BoldFont    = *Bold
]{XITS-Math}


%% --------------------------------------
%% Данные для генерации титульного листа
%% --------------------------------------
\filltitle{
    % сейчас очень часто стали меняться требования к титульному листу,
    % поэтому добавлена вспомогательная команда, которая подгрузит
    % указанный PDF (имя должно быть на английском, если Windows) 
    % и поместит его вместо шаблонного титульного листа
    % параметры ниже (title, type, ...) игнорируются
    % titlepage           = {front.pdf},
    
    title              = {\uppercase{очень интересная работа с очень длинным названием, которое не помещается в одной строке}},
    type               = {bachelor},   % bachelor, master, coursework
    sex                = {male},       % male, female
    course             = 4,
    author             = {А.\,С. Пупкин},
    authorgenitive     = {А.\,С. Пупкина},
    supervisorPosition = {д.\,ф.-м.\,н., профессор},
    supervisor         = {А.\,А. Профессор},
    chairHead          = {В.\,С. Пилиди},

    % Параметры по-умолчанию, которые с вероятностью 99% вам не стоит переопределять
    % university          = {Федеральное государственное автономное образовательное\\учреждение высшего образования\\ \vspace*{0.1cm}\uppercase{Южный Федеральный Университет}},
    % faculty             = {Институт математики, механики и компьютерных наук\\ имени И. И. Воровича},
    % chair               = {Направление подготовки\\02.03.02 -- Фундаментальная информатика\\и информационные технологии},
    % reviewer            = {reviewer},
    % reviewerPosition    = {reviewerPosition},
    % city                = {Ростов-на-Дону},
    % year                = {\the\year},
}

\begin{document}
    \maketitle
    % !TEX root = ../diploma.tex

\section*{Постановка задачи}

Ваша постановка задачи.

    \tableofcontents

    % !TEX root = ../diploma.tex

\section*{Введение}\addcontentsline{toc}{section}{Введение}

Описание требований к работе можно найти по \href{http://it.mmcs.sfedu.ru/docs/IT-papers-2015.pdf}{ссылке}.



    % !TEX root = ../diploma.tex

\section{Цитирование и ссылки}
В \LaTeX~присутствует два механизма указания ссылок: это метки и цитирование. 

\paragraph{Ссылки} указываются на произвольную метку, который задает сам пользователь. Для установки используется \verb+\label{<маркер>:<имя>}+. Ссылками можно помечать заголовки и любые окружения. 

Маркер совершенно не обязательно указывать, но его наличие сильно упрощает управление и поиск ссылок. 

Ссылаться на метку можно при помощи команды \verb+\ref{<имя>}+.

Существует несколько вариантов получения ссылки. Одним из таких специальных вариантов является \verb+\eqref{<имя>}+. Разница выглядит следующим образом:
\begin{equation}\label{eq:1}
    a^2+b^2=c^2, 
\end{equation}
\verb+\eqref{eq:1} и \ref{eq:1}+: \eqref{eq:1} и \ref{eq:1}.

Существуют и другие варианты, например, \verb+\autoref{eq:1}+: \autoref{eq:1}.

\paragraph{Цитирование} происходит на элемент в библиографии. Для цитирования используется следующая команда: \verb+\cite{<ссылка>}+. 

Любые ссылки не пишутся слитно, поэтому перед \verb+\cite+ нужен пробел. Выглядит это так: \cite{test}.

Как и со ссылками, существуют отличные от \verb+\cite+ команды цитирования. Их достаточно много, с кучей тонкостей, и обсуждение вопросов оформления цитирования выходит за рамки обзора возможностей \LaTeX.



    % !TEX root = ../diploma.tex

\section{Списки}

Существует 3 базовых окружения для списков.

Маркированные списки:
\begin{itemize}
    \item пункт 1
    \item пункт 2
\end{itemize}

Нумерованные списки:
\begin{enumerate}
    \item пункт 1
    \item пункт 2
\end{enumerate}

Описания:
\begin{description}
    \item[Описание] текст
    \item[Описание] текст
\end{description}

Иногда хочется сжать список. Чтобы не настраивать интервал между списками (это делается не очень удобно) достаточно передать параметр \texttt{[noitemsep]}. 

Этот параметр (а также ряд других, которые можно посмотреть в документации) можно задать глобально после подключения пакета \texttt{enumitem}: \verb+\setlist[itemize]{noitemsep, nolistsep}+.


Без сжатия:
\begin{itemize}
    \item пункт 1
    \item пункт 2
\end{itemize}

Со сжатием:
\begin{itemize}[noitemsep]
    \item пункт 1
    \item пункт 2
\end{itemize}

    % !TEX root = ../diploma.tex

\section{Код и псевдокод}
Вставить код тоже просто. Если настройки листинга не устраивают, их можно изменить. Макрос настройки находится в файле \texttt{commands}.

\begin{lstlisting}[caption={Пример вызова БПФ в библиотеке \texttt{CuFFT}}]
    cufftComplex *d_signal;
    cudaMalloc((void **) &d_signal, mem_size);
    cudaMemcpy(d_signal, fg, mem_size, cudaMemcpyHostToDevice);

    cufftHandle plan;
    cufftPlan2d(&plan, N, N, CUFFT_C2C);

    cufftExecC2C(plan, (cufftComplex *)d_signal, (cufftComplex *)d_signal, CUFFT_FORWARD);
\end{lstlisting}

Также можно писать псевдокод. Ключевые слова можно переводить, вводить новые конструкции и т.д. Пример в файле \texttt{commands}.
\begin{algorithm}[H]
    \caption{Пример псевдокода}
    \begin{algorithmic}[1] % The number tells where the line numbering should start
        \Procedure{F}{$A$, $B$, $N$}
        \State $E \gets A$
        \For{$i := 1$ до $N$}
        \State $\hat{E} = \texttt{fft}~E$
        \EndFor
        \State \textbf{вернуть} $E$
        \EndProcedure
    \end{algorithmic}
\end{algorithm}

    % !TEX root = ../diploma.tex

\section{Таблицы}
Здесь используется вспомогательное окружение \texttt{tabularx} (а также симметричное к нему \texttt{tabulary}), которое управляет шириной столбцов и автоматически переносит текст на новую строку в той же ячейке при нехватке размерности, доступной тексту.

\begin{table}[H]
    \centering
    \begin{tabularx}{\textwidth}{| X | X | X |}
    	\hline
    	\textbf{Размер изображения} & \textbf{Время GPU} & \textbf{Время CPU} \\ \hline
    	$1920\times 1920$           & $6$ мс             & $75$ мс            \\ \hline
    	$4096\times 4096$           & $24$ мс            & $520$ мс           \\ \hline
    	$3648\times 5472$           & $35$ мс            & $625$ мс           \\ \hline
    \end{tabularx}
    \caption{Сравнение скорости работы}
\end{table}

Таблицы как выше просты в использовании, но их внешний вид оставляет желать лучшего. При оформлении работ можно руководствоваться следующим {\href{https://www.inf.ethz.ch/personal/markusp/teaching/guides/guide-tables.pdf}{\textbf{пособием}}.


\begin{table}[H]
    \centering
    \caption{\label{tab:widgets}Подпись к таблице --- сверху}
    \begin{tabular}{llr}
    	\toprule
    	          \multicolumn{2}{c}{Item}            &           \\
    	\cmidrule(r){1-2}
        Животное & Описание & Цена (\$) \\ \midrule
    	Armadillo                          & frozen   &      8.99 \\ \bottomrule
    \end{tabular}
\end{table}

    % !TEX root = ../diploma.tex

\section{Фигуры}
В окружение \texttt{figure} можно помещать обычный \texttt{includegraphics}, таблицы, элементы \texttt{tikz}, создавать массивы изображений и т.д.

\paragraph{Пример массива изображений.}
Подписи не обязательны. Нумерацию \\ \texttt{subfloat}'ов можно выключить в \texttt{captionsetup}. Там же находится набор других настроек внешнего вида подписей.

Расстояние между картинками задается стандартными макросами шага: \texttt{quad, qquad} и т.д.

Если картинки нет, но необходимо уже сейчас отрегулировать внешний вид и размер, то можно использовать стандартные \texttt{example-image-[a,b,c]}.

\begin{figure}[H]
    \centering
    \captionsetup[subfigure]{justification=centering}
    \subfloat[Пример]{\includegraphics[height=0.1\textwidth]{example-image-a}}%
    \quad
    \subfloat[Пример]{\includegraphics[height=0.15\textwidth]{example-image-b}}%
    \quad
    \subfloat[Пример]{\includegraphics[height=0.1\textwidth]{example-image-c}}%
    \quad
    \subfloat[Очень длинная подпись, которая перенесется на новую строку]{\includegraphics[height=0.15\textwidth]{example-image-a}}%
    \caption{Общая подпись к фигуре}
\end{figure}

    % !TEX root = ../diploma.tex

\section{Загрузка и обработка CSV файлов, построение графиков Tikz}
\newcommand{\timechart}[3]
{
    \begin{figure}[H]
        \centering
        \begin{tikzpicture}[rotate=90]
        \begin{axis}
        [           xlabel             = {Размер блока},
                    ylabel             = {Время, \SI{}{\milli\second}},
                    width              = 0.78\textheight,
                    height             = 0.9\textwidth,
                    % legend pos       = outer north east,
                    legend style       = {at = {(0.5, 0.95)}, anchor = north},
                    ymode              = log,
                    log ticks with fixed point,
                    scaled y ticks     = false,
                    y tick label style = {/pgf/number format/fixed,
                        /pgf/number format/1000 sep = \thinspace},
                    y label style      = {at = {(axis description cs: -0.09, 0.5)}, anchor = south}
        ]
        
        \addplot table [x=blockSize, y=time, col sep=comma] {\datapath/#1_#2_omp_outer.csv};
        \addlegendentry{Внешний \texttt{OpenMP}}
        
        \addplot +[restrict x to domain=6:960] table [x=blockSize, y=time, col sep=comma] {\datapath/#1_#2_omp_inner.csv};
        \addlegendentry{Внутренний \texttt{OpenMP}}
        
        \addplot table [x=blockSize, y=time, col sep=comma] {\datapath/#1_#2_future.csv};
        \addlegendentry{\texttt{std::future}}
        
        \addplot table [x=blockSize, y=time, col sep=comma] {\datapath/#1_#2_seq.csv};
        \addlegendentry{Последовательный}
        \end{axis}
        \end{tikzpicture}
        \caption{#3, тип \texttt{#2}}
    \end{figure}
}

\newcommand{\timetable}[2]
{
    \DTLloaddb{data_#1}{\datapath/#1}
    \begin{table}[H]
        \setlength{\columnsep}{1.5cm}
        \setlength{\columnseprule}{0.2pt}
        \begin{multicols}{3}
            \begin{itemize}[noitemsep]
                \DTLforeach{data_#1}{\a=blockSize,\b=time}{\item [\a] -- \b\, \SI{}{\milli\second}}
            \end{itemize}
        \end{multicols}
        \caption{#2}
    \end{table}
}

Существует множество пакетов для загрузки данных из CSV, причем все они обладают различной функциональностью и капризностью (некоторые отказываются работать в одних окружениях, другие -- в других и т.п.). Поэтому если что-то не будет работать как нужно, возможно, потребуется использовать другой пакет.

Команды для построения удобно обернуть в макросы, если планируется многократное (больше двух) использование.
\timetable{gcc_int_seq.csv}{Время работы \texttt{GCC}, последовательный, тип \texttt{int}}

\timechart{gcc}{int}{\texttt{GCC}}
    % !TEX root = ../diploma.tex

\section{Здесь пока ничего нет}

Здесь пока ничего нет
    % !TEX root = ../diploma.tex

\section*{Заключение}\addcontentsline{toc}{section}{Заключение}

Ваше заключение.


    \printbibliography[heading=bibintoc]

    % !TEX root = ../diploma.tex

\begin{appendices}
    \section{Пример приложения}
    Ваше приложение
\end{appendices}

\end{document}
